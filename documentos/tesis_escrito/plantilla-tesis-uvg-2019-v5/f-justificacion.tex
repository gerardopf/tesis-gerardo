La inteligencia de enjambre es una rama del área de inteligencia artificial. Sus aplicaciones abarcan desde realizar tareas cotidianas hasta resolver problemas complejos. Uno de sus mayores beneficios es utilizar agentes robóticos de bajo costo para ejecutar acciones en conjunto en lugar de utilizar un solo robot complejo. Además, esto crea un sistema robótico robusto lo que resulta útil para diversas aplicaciones.

Se decidió estudiar la robótica de enjambre ya que en la Universidad del Valle de Guatemala, en trabajos anteriores, se han realizado pruebas implementando algoritmos de sincronización y control utilizando simuladores como Webots, así como agentes físicos Pololu 3Pi+. Estas pruebas tuvieron éxito, sin embargo, estuvieron limitadas a generar trayectorias en un ambiente con obstáculos fijos y los tiempos de ejecución eran muy largos. Por lo tanto, en este trabajo de graduación se deseaba optimizar el algoritmo desarrollado previamente y realizar nuevas pruebas para validar las trayectorias generadas en un ambiente controlado con obstáculos móviles. 

Esto permitió conocer el alcance de los algoritmos de sincronización y control, además que abrió las puertas a su implementación en aplicaciones de la vida real como misiones de búsqueda y rescate o análisis de zonas de cultivos, donde en los escenarios se encontrarán obstáculos móviles como personas, animales o incluso cambios en el propio escenario debido a factores externos.
