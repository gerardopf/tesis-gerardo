% ===============================================================================
% El estudiante debe llenar sus datos en esta sección para que la plantilla los 
% auto-importe y genere automáticamente las páginas de portada y de firmas 
% autorizadas.
% ===============================================================================
% Datos del estudiante:
% -------------------------------------------------------------------------------
% Nombre completo
\def \nombreestudiante {Gerardo Paz Fuentes}
% Carné
\def \uvgcarne {20173}
% Facultad
\def \uvgfacultad {Ingeniería}
% Carrera
\def \uvgcarrera {Ingeniería Mecatrónica}

% Datos del trabajo:
% -------------------------------------------------------------------------------
% Título completo
\def \titulotesis {Optimización de un algoritmo de inteligencia de enjambre enfocado en sincronización y control de formaciones de sistemas robóticos multi-agente para escenarios con obstáculos móviles}
% Año de entrega
\def \anoentrega {2024}
% Asesor
\def \nombreasesor {M. Sc. Carlos Esquit}

% Datos del tribunal examinador:
% -------------------------------------------------------------------------------
% Nombre del primer examinador
\def \nombreprimerex {M. Sc. Miguel Enrique Zea Arenales}
% Nombre del segundo examinador
\def \nombresegundoex {Ing. Kurt Emmanuel Kellner}
% Año de aprobación
\def \anoaprobacion {2024}
% Mes de aprobación
\def \mesaprobacion {diciembre }
% Día de aprobación
\def \diaaprobacion {02 }

% Capítulos pre-definidos
% -------------------------------------------------------------------------------
% Comentar las líneas de las secciones que desean omitirse, por defecto se 
% se incluyen todas.
\def \CAPprefacio {Prefacio}
\def \CAPantecedentes {Antecedentes}
\def \CAPalcance {Alcance}
\def \CAPanexos {Anexos}
\def \CAPglosario {Glosario}

% Formato y estilo de la plantilla
% -------------------------------------------------------------------------------
% Modo impresión: Puede des-comentar la siguiente línea para generar un documento pdf sin la portada, para cuando se desee imprimir el documento para encuadernación
%\def \printver {Versión del documento para impresión}

% Portada: Puede cambiarse la imagen en la portada al cambiar el nombre del 
% archivo siguiente. NOTA: debe tener la suficiente resolución para cubrir el área
% designada
\def \imagenportada {plantilla/portadacit.jpg}

% Referencias: Puede des-comentar la siguiente línea para utilizar el formato de referencias APA
%\def \usarAPA {Usar formato APA}

% Párrafo: Puede comentar la siguiente línea si desea emplear un formato de 
% párrafo distinto al establecido por defecto
\def \parpordefecto {Formato de párrafo por defecto}

% Capítulos y secciones: Puede des-comentar la siguiente línea para establecer el
% formato de los capítulos y secciones bajo el estándar original de UVG para
% trabajos de graduación. Este incluye: capítulos con numeración romana, secciones
% con letras mayúsculas, sub-secciones con números y sub-sub-secciones con letras
% minúsculas
%\def \capsecuvg {Formato UVG para capítulos y secciones}