\begin{enumerate}
	\item Durante el proceso de optimización del algoritmo original, se identificó que una de las principales limitaciones radica en la comunicación TCP con el servidor del Robotat. Esto se debe a que la actualización en tiempo real de las posiciones de los marcadores depende directamente de la latencia de dicha comunicación.
	\item La optimización en la asignación de posiciones de los agentes dentro del grafo de formación, resultó en un sistema más robusto y escalable. Esto fue especialmente significativo en el caso del grafo de formación triangular, ya que facilita la implementación de mejoras que permitan expandir las formaciones para incluir más de diez agentes.
	\item La optimización del algoritmo, basada en las deficiencias identificadas, permitió desarrollar una versión más robusta y eficiente. Esta nueva versión no solo está mejor preparada para aplicaciones más demandantes, si no que también facilita la experimentación y mejora su escalabilidad.
	\item La optimización del algoritmo implementando operaciones matriciales y paralelismo computacional fue exitosa ya que resultó en una reducción significativa en los tiempos de ejecución, especialmente al incrementar el número de agentes en la formación.
	\item El diseño del algoritmo permite ajustar las trayectorias de los agentes en tiempo real, respondiendo a cambios en las posiciones de los obstáculos y evitando las colisiones.
	\item Con $26$ pruebas físicas controladas, se demostró que el algoritmo optimizado es válido para su uso en entornos dinámicos y controlados, lo que le da potencial a su aplicación en escenarios más complejos para futuras investigaciones.
	\item La modificación en la asignación de marcadores para cada agente, permitiendo el uso libre de cualquier marcador sin importar el orden de selección, facilitó significativamente la depuración y ejecución de los experimentos. Especialmente en los casos donde se tuvo que compartir los marcadores y agentes con otros compañeros.
\end{enumerate}