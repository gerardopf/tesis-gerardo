In this graduation work, the implementation of the swarm intelligence algorithm focused on synchronization and control of robotic system formations was optimized. For its development, Pololu 3Pi+ differential robots were used within the Robotat ecosystem of the Universidad del Valle de Guatemala.

The process began by verifying the performance of the original algorithm, using simulations with Webots and in physical scenarios. Then, areas of improvement in its implementation were identified, focusing on increasing scalability, robustness and computational efficiency through the use of matrix operations, parallelism, adjustments in position allocation and improving other functionalities.

As part of the validation, $24$ experiments were carried out to verity the performance of the optimized algorithm. The results showed a significant reduction in execution times, particularly as the number of agents in the formation increased. In addition, the optimized algorithm was shown to be suitable for generating trajectories in scenarios with moving obstacles. This was validated with $26$ controlled physical experiments, in which agents adjusted their trajectories in real time, responded to changes in obstacle positions and avoided collisions, confirming the effectiveness and adaptability of the algorithm.