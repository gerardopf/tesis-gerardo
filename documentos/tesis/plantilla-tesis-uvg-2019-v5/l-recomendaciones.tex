\begin{enumerate}
	\item Evaluar la migración del protocolo de comunicación TCP del servidor del Robotat a UDP para mejorar la velocidad de respuesta de los agentes y trabajar con múltiples solicitudes en paralelo con el servidor. Esto permite tener una menor latencia al trabajar con un sistema en tiempo real.
	\item Se recomienda implementar un algoritmo de planificación de trayectorias robusto, a manera de que la ecuación de consenso tome como objetivo los puntos de la trayectoria generada. Con esto se tendrá un algoritmo más eficiente y robusto para escenarios más complejos.
	\item Explorar la adaptación del algoritmo de sincronización y control de formaciones con drones en un espacio tridimensional. Esto permite evaluar su uso en situaciones más complejas como la exploración de entornos desconocidos.
	\item Evaluar la implementación de un sistema de control descentralizado con el algoritmo de sincronización y control de formaciones. Este enfoque permite que cada agente tenga la capacidad de procesar sus propias funcionalidades según lo necesite, liberando la carga computacional del sistema.
	\item Se recomienda modificar el algoritmo para adaptarse a situaciones en que se agregue o se elimine algún agente de la formación. Esto permite explorar un comportamiento similar al que tienen los animales.
\end{enumerate}