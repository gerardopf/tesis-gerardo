Con la creación del algoritmo de sincronización y control de formaciones se abrió paso a experimentar con él en diferentes escenarios. Además, la previa adaptación del algoritmo para utilizarlo tanto en simulaciones como en escenarios reales en el Robotat, permitió tener la infraestructura necesaria para seguir evaluando el alcance de este utilizando las plataformas robóticas como el Pololu 3Pi+. 

El algoritmo funciona en un ambiente cambiante gracias a la forma en que se planteó originalmente la ecuación de consenso y el cálculo del peso. Así como que en un ambiente real, se evalúa constantemente la posición actual de cada agente y se compara con la posición actual de los obstáculos para ajustar la ecuación de consenso a manera de evitar posibles colisiones.

Actualmente, se puede utilizar formaciones con un máximo de 10 agentes. Esto es debido a que el grafo de formación que utiliza el algoritmo, se planteó a manera de que fuera mínimamente rígido para únicamente 10 agentes. Sin embargo, en futuras implementaciones este se podría modificar para crear formaciones con más agentes.

El algoritmo está completamente desarrollado en Python que es un lenguaje de código abierto (\textit{open source}). Esto permite la facilidad de desarrollo para futuros investigadores, además que cuenta con librerías para mejorar la eficiencia computacional.

En este proyecto, se optimizó el algoritmo aplicando operaciones matriciales en lugar de bucles con Python para reducir el tiempo de procesamiento computacional. Esto permitió disminuir el tiempo que toma cada ciclo de ejecución del algoritmo, que es más notorio conforme se aumenta el número de agentes en la formación. Además, se modificó el código de procesamiento de datos que genera las gráficas y trayectorias para mostrar el comportamiento de los obstáculos móviles.

Una de las limitantes al trabajar en la optimización del algoritmo y realizar las pruebas en físico, fue que solo se cuenta con 8 agentes robóticos disponibles. Esto se debe a la disponibilidad de robots en la UVG y a que otros estudiantes los utilizan para sus proyectos. Además, el tiempo máximo para realizar pruebas en el Robotat fue limitado ya que fue necesario compartir la mesa de pruebas y los agentes robóticos con otros estudiantes que también los utilizaban, requiriendo todo el espacio disponible.
