El objetivo de este proyecto fue continuar indagando en la robótica de enjambre al seguir con el desarrollo e implementación de un algoritmo de sincronización y control de formaciones de sistemas robóticos multi-agente. Durante las fases previas se llegó hasta la validación del algoritmo en un entorno real utilizando el ecosistema del Robotat, sin embargo, no se profundizó en su funcionamiento al utilizar escenarios con obstáculos móviles. A pesar de esto, se dejó una buena infraestructura en cuando a código y su implementación en el Robotat para seguir explorando el alcance del algoritmo con escenarios más complejos.

El propósito de este proyecto fue optimizar la implementación del algoritmo de inteligencia de enjambre desarrollado anteriormente, y validarlo en escenarios con obstáculos móviles en el ecosistema del Robotat. Para cumplir con esto, se utilizó el sistema de captura de movimiento OptiTrack, la mesa de pruebas, los agentes robóticos Pololu 3Pi+ modificados y red de comunicación con el servidor del Robotat.

En primer lugar, se llevó a cabo el levantamiento de la fase preliminar del algoritmo, para lo cual se simularon escenarios en WebotsR2023b. Una vez que las simulaciones funcionaron correctamente, se buscó replicar el funcionamiento del algoritmo en el Robotat. Durante este proceso, se encontró con obstáculos como que se realizaron cambios en las conexiones de los Pololu 3Pi+, modificaciones en los marcadores disponibles y la necesidad de ajustar los parámetros de control. Todo este proceso se describe a detalle en el Capítulo \ref{cap:restauracion}.

Una vez replicado el funcionamiento del algoritmo en simulación y físico, se puso a prueba su naturaleza dinámica. En el Capítulo \ref{cap:naturaleta_dinamica} se explica a detalle cómo funciona y se muestran los escenarios de prueba utilizados.

El siguiente paso fue optimizar la implementación del algoritmo. Para esto, se identificaron deficiencias y puntos de mejora en el código, lenguaje de programación y métodos de comunicación. En el Capítulo \ref{cap:optimizacion} se detalla el proceso realizado y se muestran los resultados de la optimización.

Finalmente, en el Capítulo \ref{cap:validacion} se muestra la validación del algoritmo optimizado utilizando escenarios con obstáculos móviles en el Robotat.